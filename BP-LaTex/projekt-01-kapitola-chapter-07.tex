\chapter{Záver}
\section*{Zhodnotenie práce}
SCADA systémy sú v dnešnej dobe veľmi rozšírené a stále pribúdajú nové spoločnosti, ktoré ich využívajú. Avšak rovnako narastá aj počet útočníkov a hrozba, ktorej musia čeliť. Je preto veľmi potrebné a dôležité vedieť jednotlivé útoky včas detekovať a systém pred nimi chrániť. Táto práca sa zameriava na komunikačné protokoly DLMS/COSEM a IEC 104, ktoré sú využívané najmä v energetických odvetviach priemyslu. \par
Práca môže pomocť pri skúmaní reakcií systému na jednotlivé typy útokov a tiež pri vývoji rôznych typov monitorovacích zariadení a sónd, ktoré budú schopné útoky včas odhaliť pri sledovaní prebiehajúcej komunikácie. Súčasťou práce je zhodnotenie rôznych simulačných nástrojov na prevádzku SCADA systémov využívajúcich komunikačný protokol DLMS/COSEM a IEC 104. Pomocou nástrojov a novo-vytvoreného simulačného programu klienta pre jazyk C++ bolo vytvorené simulačné prostredie, umožňujúce testovať jednotlivé útoky a sledovať reakcie systému na ne. \par
Pri jednotlivých testovaniach bola vždy vytvorená jedna vzorová komunikácia a v nej následne niekoľko zmien (jedna v rámci testu). Výsledky testov a zaznamenaná komunikácia bola porovnávaná so vzorovou a následne bolo vyhodnotené chovanie systému na zmeny. \par
Posledná kapitola práce je venovaná popisu rôznych spôsobov detekcie útokov a prevencie pred nimi.
\section*{Výstup práce}
Výstupom práce sú programy a postup vytvorenia simulačného prostredia na testovanie rôznych typov útokov. Vďaka simulačnému programu klienta, ktorý pracuje na štvrtej (transportnej) vrstve TCP/IP modelu je postup aplikovateľný na širokú škálu komunikačných protokolov pracujúcich nad TCP/IP. Výstupom jednotlivých testovaní je datová sada vo formáte .pcap obsahujúca jednotlivé útoky spolu s popisom reakcií systému na ne. Všetky zdrojové kódy a zachytené komunikácie sú k dispozícií na priloženom médiu a v github repozitári\footnote{Github \url{https://github.com/janpristas/bakalarska-praca}}. \par
Práca bola taktiež prezentovaná na študentskej konferencií Excel@FIT a je súčasťou projektu IRONSTONE\footnote{Projekt Ironstone \url{http://www.fit.vutbr.cz/units/UIFS/grants/index.php.cs?id=1101}} vo fakultnej výzkumnej skupine NES@FIT.
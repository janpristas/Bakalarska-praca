\chapter{Úvod}
\label{uvod}
\section{Motivácia}
\tab Priemyslené IoT siete, tzv. SCADA systémy sú dnes veľmi využívané mnohými spoločnosťami. V Českej republike sú SCADA systémy využívané napríklad spoločnosťami ako RWE, E.ON alebo skupinou ČEZ. Útoky na takéto veľké komplexy môžu mať obrovský dopad nielen na spoločnosť ako takú, ale aj na bežných užívateľov. Napríklad útok na elektrárenský komplex môže spôsobiť výpadok prúdu pre tisíce domacností. Preto si myslím, že je v dnešnej dobe naozaj dôležité, aby boli tieto systémy dostatočne chránené, nakoľko sú veľmi často súčasťou kritickej infraštruktúry a majú veľký dopad na spoločnosť. Viď. napríklad zákon č. 181/2014 Sb. o kybernetických útokoch, ktorý radí energetiku do kritickej infraštruktúry\footnote{Zákon o kybernetických útokoch \url{https://www.zakonyprolidi.cz/cs/2014-181} [Online: Máj 2018]}.
\section{Postup práce}
\tab Hlavným cieľom tejto bakalárskej práce je naštudovať si komunikáciu pre priemyselné siete IoT a systémy SCADA so zameraním na komunikačné protokoly IEC 60870-5-104 a DLMS/COSEM. Následne je cieľom preskúmať známe útoky na tieto siete a zamerať sa na možnosti emulácie takýchto útokov. Výstupom práce je datová sada súborov vo formáte {\tt .pcap}, ktorá obsahuje typickú komunikáciu IoT sietí a komunikáciu obsahujúcu známe útoky. V poslednej časti práce sú popísané možnosti detekcie jednotlivých útokov spolu s možnosťami ochrany sietí pred nim.
\section{Rozdelenie práce}
\tab V druhej kapitole je uvedený teoretický základ sietí IoT a systémov SCADA. Kapitola popisuje princíp fungovania sietí, ich výhody a nevýhody. Ďalej nasleduje popis komunikácie v sieťach so zameraním na protokoly IEC 60870-5-104 a DLMS/COSEM. V tretej kapitole sú podrobne popísané existujúce nástroje na simulovanie a emuláciu prevádzky SCADA systémov. Programy komunikujú pomocou vyššie spomínaných protokolov. Pri každom sú uvedené stručné informácie o výrobcovi, popis nástroja, potrebné zariadenia pre správnu funkcionalitu, typ topológie, ktorú umožňujú vytvoriť a prípadová štúdia, ktorá popisuje ako bol daný nástroj testovaný. Bola vytvorená sada {\tt .pcap} súborov, ktoré sú uložené v Github repozitári. Na konci kapitoly je vzájomné porovnanie jednotlivých programov a vyhodnotenie ich použiteľnosti pre účely tejto práce. \par
Štvrtá kapitola popisuje bezpečnosť v priemyselných sieťach IoT a je v nej uvedených niekoľko zaznamenaných útokov na takéto siete z dôvodu ilustrácie možného dopadu na okolie a bežných občanov. \par
Piata kapitola je rozdelená na dve časti. V prvej, je uvedený podrobný popis vytvárania emulačného prostredia na testovanie rôznych typov útokov. Druhá časť je zameraná na popis testovania reakcií systému na rôzne druhy narušenia a na sledovanie reakcií systému na ne.
V poslednej kapitole sú preberané rôzne postupy monitorovania sietí a detekcie jednotlivých typov útokov. Útoky sú rozdelené do niekoľkých kategórií na základe spôsobu prieniku do systému a jeho ohrozenia. 
\section{Prínos práce}
\tab Práca obsahuje popis a testovanie dostupných emulačných nástrojov priemyselnej komunikácie spolu s popisom komunikačných protokolov využívaných v systémoch SCADA. Z poznatkov o protokoloch bol vytvorený nástroj umožňujúci emulovať rôzne typy útokov na priemyselné siete. Pomocou môjho nástroja a dostupných emulačných nástrojov bolo vytvorené testovacie prostredie na testovanie jednotlivých typov útokov a na sledovanie reakcií systému na ne. Výstupom práce je vygenerovaný .pcap dataset obsahujúci rôzne typy útokov a narušení. Jednotlivé .pcap súbory spolu s emulačným nástrojom sú k dispozícií v github repozitáry\footnote{GitHub \url{https://github.com/janpristas/bakalarska-praca}}. \par
Výsledky tejto práce môžu taktiež pomocť mnohým spoločnostiam predísť nežiadúcim ohrozeniam ich systémov a efektívne sa brániť väčšine typov známych útokov na ich siete. Výstupy práce, konkrétne datová sada .pcap súborov, je súčasťou projektu IRONSTONE\footnote{Projekt Ironstone \url{http://www.fit.vutbr.cz/units/UIFS/grants/index.php.cs?id=1101}} vo výzkumnej skupine NES@FIT.